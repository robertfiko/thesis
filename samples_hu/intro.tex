\chapter{Bevezetés}
\label{ch:intro}

\section{Statikus kódelemzés, illetve refaktorálás}

Mi az a kód elemzés? Nem a programozó feladata a kódot megérteni? Ezen kérdések mind felmerülhetnek bennünk. A kódelemező egy olyan szoftver, ami képes a forráskód elemzésére, annak futtatása nélkül. Ugyan egy kicsi projektben egy elemző program használata még nem feltétlen szükséges, azonban egy ipari méretű szoftver esetében a programozók feladatát nagyban megkönnyítik.

Egy statikus elemző szoftver segítségével rengeteg információt tudhatunk meg még a fejlesztés során, mint például: \textit{kód-metrikákat} és \textit{függőségi információkat}, melyek olyan kódrészek tekintében felettébb fontosak, amiket nem mi írtunk.  

Szintén nagy méretű projekteknél fontos a refaktoráló, azaz kód átalakító eszközök használata is, hiszen manuálisan egy függvény vagy egy változó átnevezése igen nagy kihívást jelentene. Azonban nem csak ilyen méretű szoftvereknél hasznos, amit esetlegesen több ember is használt, hanem egy projekt fejlesztésében, fejlődésében is jelentősége van egy ilyen, refaktoráló eszköznek. Gondoljunk bele: elkezdünk egy projektet, (ami lehet egy független szoftver, vagy akár egy új komponens egy már meglévő projektben) először egy prototípust készítünk belőle, majd az alapján folytatjuk a fejlesztést, jellemzően már egy tervet követve, azonban a legalaposabb tervezés ellenére is előfordulhat, hogy egy kódot át kell alakítani.

\section{RefactorErl}
A RefactorErl az ELTE Informatikai Karán futó kutatás-fejlesztési projekt melynek fő célja az, hogy Erlang forrásokhoz statikus elemző szoftvert készítsen. Az eszközzel az elemzés mellett kódátalakítást, refaktorálást is elvégezhezünk. Főbb funkcionalitásaiként kiemelném a: függőségi gráf vizualizációját, illetve a szemantikus lekérdező nyelvét, amelyen keresztül többek között olyan információk érhetőek el, mint nem használt makró definícók, változó lehetséges értékei vagy biztonsági sérülékenységek \cite{refactorerlSecurity}. Az eszköznek jelenleg több felhasználói felülete van, amelyből kiemelném a webes felületet és a Erlang Shell-en keresztül elérhető parancssori felületét.

\section{Fejlesztő környezeti nyelvi támogatás}
Napjainkban nagyon elterjedtek az integrált fejlesztői környezetek, avagy IDE\footnote{\textit{Integrated Developement Environment}}-k. Az ilyen fejlesztői környzetben egy szoftveren belül tudjuk szerkeszteni a forráskódot, konzolt kezelni, verziót kezelni, s a kódunkról diagnosztikákat is kapunk, szinte valós időben, amennyiben az adott programozási nyelvhez rendelkezésre áll ilyen IDE-kompatibilis rendszer. (\textit{Például: Visual Studio, IntelliJ, Emacs})

A nyelvi támogatás nem jelent mást, mint egy a fejlesztői környzetbe integrált szoftvert, ami segíti a programozó munkáját a hibák és figyelmeztetések vizualizációjában, kód-kiegészítési javaslatokat is tesz, illetve a dokumentáció egy részét is elérhetővé teszi a szerkesztőn belül (\textit{Például: az adott függvény paraméterezése, leírása})

Sok esetben az ilyen nyelvi támogató rendszerek kifejezetten egy rendszerhez készültek, egy másik szerkesztőben pedig nem használhatóak, pedig a forrást ugyanúgy kell elemezni és a fejlesztőnek is közel azonos diagnosztikákat, visszajelzéseket kell nyújtani. Erre ad megoldást a Microsoft Language Server Protocolja (LSP), ami egy fejlesztői környezet független nyelvi kiszolgáló protokollt jellemez. Ennek az előnye, hogy a kiszolgálót csak egyszer kell megírni, és onnantól kezdve könnyedén integrálható, bármely kliensbe (amennyiben az megvalósítja a protokollt)

\section{Erlang LS}
Az Erlang Language Server (röviden: Erlang LS) a nyelvi kiszolgáló az Erlang progrmaozási nyelvhez. \textit{Roberto Aloi} vezetésével indult el a nyílt forráskódú projekt, aminek mára már jelentős karbantartói és fejlesztői csapata van, a világ minden pontjáról, Magyarországról is. Az Erlang LS alapvetően egy szerkesztő független megvalósítás, azonban a csapat a Visual Studio Code-hoz fejlesztett egy interfész kiegészítőt is, ami segítségével az LS, szépen betud épülni a szerkesztőbe. Az elemző eszköz elérhető továbbá például Emacs, IntelliJ és még sok más IDE-ben is. Az Erlang LS támogotja a DAP\footnote{Debugger Adapter Protocoll}-ot és a legfrissebb kiadása már a RefactorErl diagnosztikák integrációját is támogtja.

\section{Visual Studio Code}
A Visual Studio Code manapság az egyik legelterjetebb kódszerkesztő szoftver, ami köszönhető nyílt forrásának, illetve annak, hogy elérhető macOS, Linux és Windows operációs rendszereken, sőt akár (béta verzióban) az interneten is. Az eszköz beépített grafikus verziókezelést és integrált terminált is biztosít, a plugin\footnote{kiegészítő, beépülő modul} architektúrája végett pedig szinte a kedvünkre bővíthető és személyre szabható\footnote{Ezen kiegészítőket az alkalmazáson belül elérhető Marketplaceből tudjuk letölteni.}.

\section{Feladat}

A feladat egy olyan alkalmazás csomagot készíteni, melynek komponensei egymással képesek együtt műküdni, s ezálltal az Erlang nyelven fejlesztők munkáját megkönnyíteni. A funckionalitásokat két felé kell csoportosítani: amiket meglehet valósítani az Erlang LS-ben és amiket nem. Sajnos amiket nem tudunk megvalósítani a Language Server Protocoll felett, azok csak a Visual Studio Code szerkesztőben fognak működni, egy önálló bővítmény segítségével.

Az alkalamazás jelenítsen meg diagnosztikákat és kód vizualizációkat a forrásról. Lehessen egyszerűen függőségi információkat kinyerni modul és függvény szinten. Továbbá legyen lehetőség beépített és egyedi lekérdezések futtatására, illetve azok megjelenítésére.