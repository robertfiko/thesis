\chapter{Fejlesztői dokumentáció}
\label{ch:impl}


\section{Megoldandó feladat}
\section{Használt eszközök}
\subsection{RefactorErl}
\subsection{Az Erlang nyelv}
\subsection{Erlang LS}
\subsection{A TypeScirpt nyelv}
\subsection{Visual Studioc Code API}

\section{Komponensek viszonya egymáshoz}
Erlang LS

RefactorErl

Visualiser

RPC

Websocket

\section{LSP fölött megvalósított funkcionalitások}
\section{Bővítménybe ágyazott funkcionalitások}















\section{Forráskódok}


\lstset{caption={Hello World in C++}, label=src:cpp}
\begin{lstlisting}[language={C++}]
#include <stdio>

int main() 
{
	int c;
	std::cout << "Hello World!" << std::endl;

	std::cout << "Press any key to exit." << std::endl;
	std::cin >> c;
	
	return 0;
}
\end{lstlisting}


\subsection{Algoritmusok}

Az \ref{alg:ibb}.~algoritmus egy általános elágazás és korlátozás algoritmust (\emph{Branch and Bound algorithm}) mutat be. A \ref{step:selrule}.~lépésben egy megfelelő kiválasztási szabályt kell alkalmazni.
Példa forrása: \href{https://www.inf.u-szeged.hu/actacybernetica/}{Acta Cybernetica (ez egy hiperlink)}.

\begin{algorithm}[H]
\caption{A general interval B\&B algorithm}
\label{alg:ibb}
\textbf{\underline{Funct}} IBB($S,f$)
\begin{algorithmic}[1] % sorszámok megjelenítése minden n. sor előtt, most n = 1
\State Set the working list ${\cal L}_W$ := $\{S\}$ and the final list ${\cal L}_Q$ := $\{\}$
\While{( ${\cal L}_W \neq \emptyset$ )} \label{alg:igoend}
	\State Select an interval $X$ from ${\cal L}_W$ \label{step:selrule}\Comment{Selection rule}
	\State Compute $lbf(X)$ \Comment{Bounding rule}
	\If{$X$ cannot be eliminated} \Comment{Elimination rule}
		\State Divide $X$ into $X^j,\ j=1,\dots, p$, subintervals   \Comment{Division rule}
		\For{$j=1,\ldots,p$}
			\If{$X^j$ satisfies the termination criterion} \Comment{Termination rule}
				\State Store $X^j$ in ${\cal L}_W$
			\Else
				\State Store $X^j$ in ${\cal L}_W$
			\EndIf
		\EndFor
	\EndIf
\EndWhile
\State \textbf{return} ${\cal L}_Q$
\end{algorithmic}
\end{algorithm}
