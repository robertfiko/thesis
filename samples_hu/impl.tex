

\chapter{Fejlesztői dokumentáció}
\label{ch:impl}


\section{Megoldandó feladat}
\section{Használt eszközök és környezetek}
\subsection{Erlang nyelv}
\subsection{RefactorErl}
\subsection{Erlang LS}
\subsection{TypeScirpt}
\subsection{Visual Studioc Code API}
\subsection{WebSocket}
\subsection{Remote Procedure Calls}

\section{Komponensek}
\subsection{Erlang LS-beli illesztő modul}
\subsubsection{Bevezetés}
\subsubsection{Modul felépítés}
\subsubsection{Kommunikáció a többi komponenssel}

\subsection{RefactorErl-beli LS modul}
\subsubsection{Bevezetés}
\subsubsection{Modul felépítés}
\subsubsection{Kommunikáció a többi komponenssel}

\subsection{RefactorErl-beli WebSocket modul}
\subsubsection{Bevezetés}
\subsubsection{Modul felépítés}
\subsubsection{Kommunikáció a többi komponenssel}


\subsection{RefactorErl Visualiser kiegészítő}
\subsubsection{Bevezetés}
\subsubsection{Kiegészítő felépítés}
\subsubsection{Kommunikáció a többi komponenssel}

\section{Komponensek telepítőinek előállítása (?)}














\section{Forráskódok}


\lstset{caption={Hello World in Erlang}, label=src:erlang}
\begin{lstlisting}[language={C++}]
hello() -> io:format("Hello World!").
\end{lstlisting}

