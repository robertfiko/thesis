\chapter{Összegzés}
\label{ch:sum}


A szakdolgozatomban integráltam az RefactorErl elemzői, diagnosztikai képességeit az Erlang LS kiszolgálóba, így azokat szélesebb körben elérhetővé téve. Ezen kívül elkészítettem egy újabb felhasználói felületet a RefactorErl számára. Ebben az új felületben lehetőség van egyenesen a szerkesztőből, a kódbázisból beépített lekérdezéseket, illetve egyedi szemantikus lekérdezéseket is adni, melyek eredményét szintén a forráskkódban láthatjuk, az egyszerű fa nézetű listában. A függőségi gráffal kapcsolatos információkat szintén a szerkesztőből, vagy kódakciók segítségével akár a kódbázisból is kérhetünk, s az eredményt is megtudjuk tekinteni a szoftveren belülről.


A projektet véleményem szerint sikerült értékesen zárni. Az elkészült szoftverkomponensek harmóniában működnek együtt, ezzel segítve az Erlang programozók munkáját. Az elkészült program egy hiányt tölt be olyan tekintetben, hogy a már elérhető elemező eszközök képességeit ötvözve valami hasznosabbat kapjunk. Úgy gondolom, hogy az, hogy RefactorErl-es interfész modulok már az Erlang LS legújabb verziójában benne vannak, ezt csak alátámasszák.

\section{Továbbfejlesztési lehetőségek}

A függőségi gráfoknál jelenleg csak a szöveges és SVG alapú megjelenítés támogatott, azonban meg lehetne oldalni, hogy legyen lehetőség reszponzív okosgráfok megjelenítésére is (\textit{Smart Graph}). 

Az Erlang LS oldalán akár mégtöbb beépített diagnosztikát is lehetne adni, továbbá a kódakciók tárházában is el lehet merülni, hiszen például refaktoráló akciókat nem is implementáltam. Az ELS tekintetében még egy \textit{plugin architektúra} kifejlesztése lehetőséget biztosítana arra, hogy harmadik féltől származó diagnosztikai kiszolgálók adatot tudjanak közleni egyszerűen az ELS-be. 