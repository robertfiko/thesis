\chapter{Felhasználói dokumentáció}
\label{ch:user}

\section{Rendszerkövetelmények}
\todo{Ezt hogyan?}
\section{Telepítés}
\subsection{Erlang/OTP telepítése}

Mind a RefactorErl futtatásához, mind az Erlang LS futtatéséhoz szükségünk van az Erlang virtuális gép telepítéséhez. Ezt macOS operációs rendszer legegyszerűbben a Homebrew \cite{brewErlang} \todo{Cite} nevű csomagkezelővel tehetjük meg. Linux rendszerhez az Erlang \cite{erlangDownloads} \todo{cite}


\subsection{Visual Studio Code telepítése}
\subsection{Erlang LS bővített változatának telepítése}
\subsection{RefactorErl telepítése}
\subsection{...}
\subsection{RefactorErl Visualiser telepítése}

\subsection{}